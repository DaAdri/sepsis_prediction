\documentclass[a4paper,12pt]{article}
%\usepackage[ansinew]{inputenc}
\usepackage[french]{babel}
\usepackage[T1]{fontenc}
\usepackage[margin=1in]{geometry}
\usepackage{booktabs}
\usepackage{longtable}
\usepackage{tabularx}
\usepackage{graphicx}
\usepackage{array}
\usepackage{multirow}

%Générer une liste des abréviations groupée
\usepackage{amssymb}
\usepackage{nomencl}
\usepackage{etoolbox}
\usepackage{siunitx}
\usepackage{hyperref}
%\usepackage{biblatex} %Imports biblatex package
%\addbibresource{bibliographie.bib} %Import the bibliography file

\title{rapport_stage}
\author{sifax ziani}
\date{May 2024}

\setlength{\parskip}{1em}  % Laisser un espace entre les paragraphes
\setlength{\parindent}{0em}  % Ne pas indenter les débuts des paragraphes

\title{Rapport de Stage : Prédiction des Sepsis par des modèles d'IA}
\author{Sifax ZIANI}
\date{}

%% This will add the subgroups
%----------------------------------------------
\makenomenclature
\renewcommand\nomgroup[1]{%
  \item[\bfseries
  \ifstrequal{#1}{M}{Constantes Médicales}{%
  \ifstrequal{#1}{O}{Autres abréviations}}
]}
%----------------------------------------------

%% This will add the units
%----------------------------------------------
\newcommand{\nomunit}[1]{%
\renewcommand{\nomentryend}{\hspace*{\fill}#1}}
%----------------------------------------------

\renewcommand{\nomname}{Liste des abréviations} % titre de la liste

% Définition d'un nouveau type de colonne centré et extensible
%\newcolumntype{Y}{>{\centering\arraybackslash}X}
\DeclareUnicodeCharacter{0624}{ }
\begin{document}

\selectlanguage{french} 

\begin{center}
	\huge{Rapport de Stage Master IIA} \\[11pt]
	\Large{Prédiction de Sepsis par des modèles d'IA}
\end{center}

\rule{\textwidth}{0.5pt}
\begin{abstract}
    Text du résumé
\end{abstract}
\rule{\textwidth}{0.5pt}

\newpage
\tableofcontents
\newpage

\section{Introduction}
    Text de l'introduction ....
\newpage
	
\section{Survole du sujet}
\subsection{Sepsis}
Le sepsis peut être défini comme un dysfoncionnement des organes potentiellement mortelle causée par une réponse inapropriée du corps à une infection. Pour une opérationnalisation clinique, le dysfoncionnement des organes peut être représenté par une augmentation de 2 points ou plus du score Sequential [Sepsis-related] Organ Failure Assessment (SOFA), qui est associée à une mortalité en milieu hospitalier supérieure à 10 \%. Le choc septique devrait être défini comme une sous-catégorie du sepsis dans laquelle des anomalies circulatoires, cellulaires et métaboliques particulièrement profondes sont associées à un risque de mortalité plus élevé que le sepsis seul.

Les patients atteints de choc septique peuvent être cliniquement identifiés par un besoin en vasopresseurs pour maintenir une pression artérielle moyenne de 65 mm Hg ou plus et un taux de lactate sérique supérieur à 2 mmol/L (> 18 mg/dL) en l'absence d'hypovolémie. Cette combinaison est associée à des taux de mortalité hospitalière supérieurs à 40 \%. Dans les contextes extra-hospitaliers, aux urgences ou dans les services hospitaliers généraux, les patients adultes suspectés d'infection peuvent être rapidement identifiés comme étant plus susceptibles d'avoir des résultats défavorables typiques du sepsis s'ils présentent au moins 2 des critères cliniques suivants qui constituent ensemble un nouveau score clinique au chevet appelé quickSOFA (qSOFA) : un rythme respiratoire de 22/min ou plus, une altération de l'état mental ou une pression artérielle systolique de 100 mm Hg ou moins. \cite{bone1992definitions}
\subsubsection{Concepts clés sur le Sepsis}
\begin{itemize}
	\setlength{\itemsep}{1pt}%
	\item Le sepsis est la principale cause de décès par infection, notamment s'il n'est pas reconnu et traité rapidement. Sa reconnaissance exige une attention urgente.
	\item Le sepsis est un syndrome façonné par des facteurs pathogènes et des facteurs hôtes (par exemple, le sexe, la race et autres déterminants génétiques, l'âge, les comorbidités, l'environnement) dont les caractéristiques évoluent avec le temps. Ce qui différencie le sepsis d'une infection est une réponse de l'hôte aberrante ou déséquilibrée et la présence de dysfonctionnement organique.
	\item Le dysfonctionnement organique induit par le sepsis peut être occulte ; par conséquent, sa présence devrait être envisagée chez tout patient présentant une infection. Inversement, une infection non reconnue peut être la cause d'un dysfonctionnement organique de nouvelle apparition. Tout dysfonctionnement organique inexpliqué devrait donc éveiller la possibilité d'une infection sous-jacente.
	\item Le phénotype clinique et biologique du sepsis peut être modifié par une maladie aiguë préexistante, des comorbidités de longue date, des médicaments et des interventions.
	\item Certaines infections peuvent entraîner un dysfonctionnement organique local sans générer de réponse hôte systémique déséquilibrée.
\end{itemize}

\subsection{SIRS }
.............. à remplir .......................\\

\subsubsection{SOFA (Organ Failure Assessment Score)}
Le score SOFA a été développé pour évaluer la morbidité aiguë des maladies critiques au niveau de la population et a été largement validé comme un outil à cette fin dans une gamme de paramètres de santé et d'environnements.
Ces dernières années, le score SOFA est devenu largement utilisé dans une gamme d'autres applications. Un changement de 2 points ou plus dans le score SOFA est désormais une caractéristique définissante du syndrome septique, et l'Agence européenne des médicaments a accepté qu'un changement dans le score SOFA soit un marqueur de substitution acceptable de l'efficacité dans les essais exploratoires d'agents thérapeutiques novateurs dans la septicémie. \cite{lambden2019sofa} \\ 

En résumé, le score SOFA est un système de notation destiné à déterminer l'étendue de la fonction organique d'une personne ou le taux de défaillance selon six scores différents, qui représentent respectivement les systèmes respiratoire, cardiovasculaire (avec ajustement pour les agents inotropes), hépatique, de coagulation, rénal et neurologique, chacun noté de 0 à 4, un score croissant reflétant une dysfonction organique croissante, comme détaillé dans le tableau \ref{tab:score_sofa} en page \pageref{tab:score_sofa}.\\

\subsection{qSOFA \cite{singer2016third}}
Bien que le qSOFA (pour quick SOFA) soit moins robuste qu'un score SOFA. Cependant, pour une évaluation rapide et ne nécessitant pas de tests de laboratoire, la nouvelle mesure qSOFA qui intègre seulement une altération de l'état mental, une pression artérielle systolique de 100 mm Hg ou moins, et une fréquence respiratoire de 22/min ou plus, est utilisée pour inciter les cliniciens à enquêter davantage sur une possible dysfonction organique, à initier ou intensifier le traitement approprié, et à envisager une orientation vers les soins intensifs ou à augmenter la fréquence de la surveillance, si ces actions n'ont pas déjà été entreprises.\\ 

\subsection{SIRS \cite{bone1992definitions}}
Deux ou plus de ces 4 mesures : 
Temperature > 38°C or < 36°C
Heart rate > 90/min
Respiratory rate > 20/min or Paco2 < 32 mm Hg (4.3 kPa)
White blood cell count > 12000/mm3 or < 4000/mm3 or > 10 \\ % immature bands

\subsection{SAPS II \cite{le1993new}}
Le score SAPS II a été développé en 1993 \cite{le1993new} et est un système de classification de la gravité de certaines maladies, utilisé, comme pour APACHE, pour estimer le risque de mortalité ou le temps de séjour dans les applications cliniques.\\
Le score SAPS II comprend 17 variables : 12 variables physiologiques, l'âge, le type d'admission (chirurgicale programmée, chirurgicale non programmée ou médicale), et trois variables liées aux maladies sous-jacentes (syndrome d'immunodéficience acquise, cancer métastatique et hémopathie maligne).

\subsection{APACHE IV \cite{zimmerman2006acute}}
La première génération du score APACHE a été développée en 1981 \cite{wa1981apache}, puis révisée et publiée en 1985 et 1991, donnant naissance respectivement à APACHE II et APACHE III \cite{knaus1985apache} \cite{knaus1991apache}. Le score APACHE IV, le plus récent des scores APACHE, a été introduit en 2006 \cite{zimmerman2006acute} et est utilisé pour estimer le risque de mortalité à court terme à partir des données cliniques réelles du premier jour après l'admission, ainsi que pour prédire la durée du séjour en unité de soins intensifs (ICU) \cite{zimmerman2006intensive}.\\
Les variables utilisées pour le calcule du score APACHE II, d'après (https://clincalc.com/IcuMortality/APACHEII.aspx) sont les suivantes : Age (years), Glasgow coma score (valeur min en 24h), Température (°C ou F), MAP (mmHg), Heart rate (bpm), Resp rate (bpm), FiO2 (\%), PaO2 (mmHg), Arterial pH, Sodium (mEq/L), Potassium (mEq/L), Creatinine (mg/dL), Acute renal failure (No/Yes), Hematocrit (\%), WBC (x 10\(^9\)/L), Severe organ system insufficiency or is immunocompromised (no/yes).
		 				
\subsection*{Synthèse}
D'après les résultats des études, dont certaines sont citées dans les 3 sections précédentes, le score sofa est le meilleur parmi les autres pour la prédiction du Sepsis (ou autres pathologies qui touchent les organes), tandis que SAPS et APACHE mesurent, en quelque sorte, la gravité de la maladie une fois celle-ci déjà détectée.

\renewcommand{\arraystretch}{1.5} % Ajuste l'espacement vertical entre les lignes

% Score SOFA 
\begin{table}[htbp]
\centering
\caption{\normalsize{Organ Dysfunction or Failure Assessment Score}}
\label{tab:score_sofa}
\begin{tabularx}{\textwidth}{|p{3cm}|X|X|X|X|X|}
\hline
\textbf{System} & \textbf{0} & \textbf{1} & \textbf{2} & \textbf{3} & \textbf{4} \\
\hline
\textbf{Respiration} & PaO\textsubscript{2}/FiO\textsubscript{2}, mm Hg $>$ 400 (53.3) & $<$ 400 (53.3) & $<$ 300 (40) & $<$ 200 (26.7) with respiratory support & $<$ 100 (13.3) with respiratory support \\
\hline
\textbf{Coagulation} & Platelets, $\times 10^3$/µL $\geq$ 150 & $<$ 150 & $<$ 100 & $<$ 50 & $<$ 20 \\
\hline
\textbf{Liver} & Bilirubin, mg/dL (µmol/L) $<$ 1.2 (20) & 1.2–1.9 (20–32) & 2.0–5.9 (33–101) & 6.0–11.9 (102–204) & $>$ 12.0 (204) \\
\hline
\textbf{Cardiovascular} & MAP $\geq$ 70 mmHg & MAP $<$ 70 mmHg & Dopamine $\leq$ 5 or dobutamine (any dose) & Dopamine 5.1–15 or epinephrine $\leq$ 0.1 or norepinephrine $\leq$ 0.1 & Dopamine $>$ 15 or epinephrine $>$ 0.1 or norepinephrine $>$ 0.1 \\
\hline
\textbf{Central nervous system} & Glasgow Coma Scale score 15 & 13–14 & 10–12 & 6–9 & $<$ 6 \\
\hline
\textbf{Renal} & Creatinine, mg/dL (µmol/L) $<$ 1.2 (110) & 1.2–1.9 (110–170) & 2.0–3.4 (171–299) & 3.5–4.9 (300–440) & $>$ 5.0 (440) \\
& Urine output, mL/d $>$ 500 & & & $<$ 500 & $<$ 200 \\
\hline
\end{tabularx}
\end{table}


\subsection{HL7 \textit{https://hl7.fr/}}
HL7 est une organisation internationale fondée en 1987, dédiée à l’interopérabilité des systèmes d’information dans le domaine de la santé.
L’implémentation des standards HL7 par différents systèmes permet d’assurer les trois niveaux d’interopérabilité indispensables au traitement efficient et sécurisé de l’information :
\begin{itemize}
    \setlength{\itemsep}{1pt} % Laisser un espace vertical entre chaque item
    \item Interopérabilité technique qui consiste à transférer les données d’un système A vers un système B,
    \item Interopérabilité sémantique qui permet au système A et au système B d’interpréter de la même façon l’information codée portée par le standard,
    \item Interopérabilité métier qui assure la cohérence entre les processus métier définis dans chaque organisation.
\end{itemize}

HL7 et ses membres fournissent un cadre (et des normes associées) pour l'échange, l'intégration, le partage et la récupération des informations de santé électroniques. Ces normes définissent comment les informations sont emballées et communiquées d'une partie à une autre, en établissant le langage, la structure et les types de données nécessaires pour une intégration fluide entre les systèmes. Les normes HL7 soutiennent la pratique clinique ainsi que la gestion, la prestation et l'évaluation des services de santé, et sont reconnues comme les plus couramment utilisées dans le monde.
\textit{ref : https://www.hl7.org/}


% Table d'évaluation des différents modèles
\renewcommand{\arraystretch}{1.5} % Ajuste l'espacement vertical entre les lignes
\begin{longtable}{|>{\small}p{2.5cm}|>{\small}p{3cm}|p{2cm}|p{1.5cm}|p{1.5cm}|p{1.5cm}|p{1.5cm}|}
\caption{Résultats expérimentaux du modèle proposé et résultats de recherche existants résumés}
\label{tab:comparaison_modeles} \\
\hline
\textbf{Auteurs} & \textbf{Jeu de données} & \textbf{Modèle} & \textbf{Temps de prédiction} & \textbf{Sensibilité} & \textbf{Spécificité} & \textbf{AUROC (95\% CI)} \\
\hline
\endfirsthead
\multicolumn{7}{c}%
{{\bfseries \tablename\ \thetable{} -- Suite de la page précédente}} \\
\hline
\textbf{Auteurs} & \textbf{Jeu de données} & \textbf{Modèle} & \textbf{Temps de prédiction} & \textbf{Sensibilité} & \textbf{Spécificité} & \textbf{AUROC (95\% CI)} \\
\hline
\endhead
\hline \multicolumn{7}{|r|}{{Suite sur la page suivante}} \\ \hline
\endfoot
\hline
\endlastfoot
Desautels et al. \cite{desautels2016prediction}, 2016 & MIMIC-III & InSight & 4 h & 0.80 & 0.54 & 0.74 \\
Kam et al. \cite{kam2017learning}, 2017 & MIMIC-II & LSTM & 3 h & 0.91 & 0.94 & 0.93 \\
Nemati et al. \cite{nemati2018interpretable}, 2018 & MIMIC-III & AISE & 4 h & 0.85 & 0.67 & 0.85 \\
Moor et al. \cite{moor2019early}, 2019 & MIMIC-III & MGP-TCN & 7 h & - & - & 0.86 \\
Scherpf et al. \cite{scherpf2019predicting}, 2019 & MIMIC-III & RNN & 3 h & 0.90 & 0.47 & 0.81 \scriptsize{(0.79–0.83)} \\
Li et al. \cite{li2020real}, 2019 & MIMIC-III & LSTM & 4 h & 0.85 & 0.60 & 0.87 \\
Shashikumar et al. \cite{shashikumar2021deepaise}, 2021 & MIMIC-III & DeepAISE & 4 h & 0.80 & 0.75 & 0.87 \\
\hline
\multirow{4}{*}{Kim et al. \cite{kim2022early}, 2022} & \multirow{4}{*}{MIMIC-III} & SOFA & 3 h & 0.65 & 0.58 & 0.63 \scriptsize{(0.59–0.67)} \\
& & qSOFA & 3 h & 0.61 & 0.75 & 0.65 \scriptsize{(0.62–0.68)} \\	
& & SAPS II & 3 h & 0.65 & 0.77 & 0.68 \scriptsize{(0.66–0.70)} \\
& & LSTM & 3 h & 0.83 & 0.74 & 0.84 \scriptsize{(0.81–0.87)} \\
\hline
\multirow{4}{*}{Kim et al. \cite{kim2022early}, 2022} & \multirow{4}{*}{MIMIC-III} & \multirow{4}{*}{NAS-GA} & 3 h & 0.93 & 0.91 & 0.94 \scriptsize{(0.92–0.96)} \\
& & & 4 h & 0.91 & 0.86 & 0.93 \scriptsize{(0.92–0.94)} \\
& & & 8 h & 0.88 & 0.82 & 0.87 \scriptsize{(0.84–0.90)} \\
& & & 12 h & 0.86 & 0.81 & 0.83 \scriptsize{(0.81–0.85)} \\	
\hline
Tang et al. \cite{tang2024time}, 2024 & eICU Collaborative Research Database & LSTM-Transformer & 12 h & 0.94 & 0.97 & 0.99\\ 
\hline
\multirow{3}{*}{Tang et al. \cite{tang2024time}, 2024 } & \multirow{3}{*}{MIMIC-III} & LSTM-Transformer & 12 h & 0.78 & 0.79 & 0.92\\
& & LSTM-Transformer & 8 h & 0.79 & 0.79 & 0.92\\
& & LSTM-Transformer & 4 h & 0.81 & 0.81 & 0.93\\
\hline
Li et al. \cite{li2019convolutional}, 2019 & PhysioNet/CinC Challenge dataset & CNN+RNN & 12 h & - & - & 0.75 \\
Rafiei et al. \cite{rafiei2021ssp},2021 & 2019 PhysioNet/CinC Challenge dataset & SSP & 4 h & 0.85 & 0.81 & 0.92 \\
Yang et al. \cite{yang2020explainable}, 2020 & PhysioNet/CinC Challenge dataset & XGBOOST & 1 h & 0.90 & 0.64 & 0.85 \\
Lauritsen et al. \cite{lauritsen2020early}, 2020 & The Danish National Patient Registry & CNN+LSTM & 3 h & - & - & 0.86 \\
Khojandi et al. \cite{khojandi2018prediction}, 2018 & Oklahoma State University & RF & 0 h & 0.99 & 0.97 & 0.90 \\
Bedoya et al. \cite{bedoya2020machine}, 2020 & Duke University Hospital & MGP-RNN & 4 h & - & - & 0.88 \scriptsize{(0.87–0.89)} \\
\end{longtable}


% tableau de synthèse des variables utilisées par les différents modèles
\begin{longtable}{|p{3.5cm}|p{1cm}|p{1cm}|p{1cm}|p{1cm}|p{1cm}|p{1cm}|p{1cm}|p{1cm}|}
    \caption{Variables utilisées par les différents modèles} \label{tab:variables_modeles} \\
    \hline
    \textbf{Variables} & \textbf{InSight} & \textbf{AISE (65 features \cite{nemati2018interpretable})} & \textbf{LSTM} & \textbf{RNN} & \textbf{MGP-TCN} & \textbf{DeepAISE} & \textbf{NAS-GA} & \textbf{LSTM-Transformer} \\
    \hline
    \endfirsthead

    \hline
    \textbf{Variables} & \textbf{InSight} & \textbf{AISE (65 features \cite{nemati2018interpretable})} & \textbf{LSTM}  & \textbf{RNN} & \textbf{MGP-TCN} & \textbf{DeepAISE} & \textbf{NAS-GA} & \textbf{LSTM-Transformer} \\
    \hline
    \endhead

    \hline
    \endfoot

    \hline
    \endlastfoot

    % Demographics
    Age & o & & o & & & & & \\
    Gender & & & o & & & & & \\
    Weight & & & o & & & & & \\
    Readmission to intensive care & & & o & & & & & \\
    Elixhauser score (premorbid status) & & & o & & & & & \\
    
    \hline
    % Vital signs
    SOFA (4h time step) & & & o & & & & & \\
    SIRS & & & o & & & & & \\
    Glasgow coma scale & & & o & & & & & \\
    Heart rate & o & & o & & o & & & \\
    systolic & o & & o & & o & & & \\
    mean and diastolic & & & o & & o & & & \\
    blood pressure & o & & o & & o & & & \\
    shock index & & & o & & & & & \\
    Respiratory rate & o & & o & & o & & & \\
    SpO2 & o & & o & & o & & & \\
    Temperature & o & & o & & o & & & \\
    Cardiac Output & & & & & o & & & \\
    Tidal Volume Set & & & & & o & & & \\
    Tidal Volume Observed & & & & & o & & & \\
    Tidal Volume Spontaneous & & & & & o & & & \\
    Peak Inspiratory Pressure & & & & & o & & & \\
    Total Peep Level & & & & & o & & & \\
    O2 flow & & & & & o & & & \\
    
    \hline
    % Laboratory values
    Potassium & & & o & & o & & & \\
    sodium & & & o & & o & & & \\
    chloride & & & o & & o & & & \\
    Glucose & & & o & & o & & & \\
    BUN & & & o & & & & & \\
    creatinine & & & o & & o & & & \\
    Creatine Kinase & & & & & o & & & \\
    Creatine Kinase MB & & & & & o & & & \\
    Magnesium & & & o & & o & & & \\
    calcium & & & o & & o & & & \\
    ionized calcium & & & o & & & & & \\
    carbon dioxide & & & o & & & & & \\
    SGOT & & & o & & & & & \\
    SGPT & & & o & & & & & \\
    albumin & & & o & & o & & & \\
    Hemoglobin & & & o & & o & & & \\
    Hematocrit & & & & & o & & & \\
    White blood cells count & o & & o & & o & & & \\
    platelets count & & & o & & o & & & \\
    Partial Thromboplastin Time & & & o & & o & & & \\
    Prothrombin Time & & & o & & o & & & \\
    International Normalized Ratio & & & o & & o & & & \\
    pH & o & & o & & o & & & \\
    PaO2 & & & o & & o & & & \\
    PaCO2 & & & o & & o & & & \\
    base excess & & & o & & & & & \\
    bicarbonate & & & o & & o & & & \\
    lactate & & & o & & o & & & \\
    Lactate Dehydrogenase & & & & & o & & & \\
    Bands (Immature Neutrophils) & & & & & o & & & \\
    Bilirubin & & & & & o & & & \\
    Blood Urea Nitrogen & & & & & o & & & \\
    Fibrinogen & & & & & o & & & \\
    SO2 & & & & & o & & & \\
    Troponin T & & & & & o & & & \\
    \hline
    % Ventilation parameters
    Mechanical ventilation & & & o & & & & & \\
    FiO2 (Fraction of Inspired Oxygen) & & & o & & o & & & \\
    
    \hline
    % Medications and fluid balance
    Current IV fluid intake over 4h & & & o & & & & & \\
    Maximum dose of vasopressor over 4h & & & o & & & & & \\
    Urine output over 4h & & & o & & & & & \\
    Cumulated fluid balance since admission & & & o & & & & & \\
    
    \hline
    % Outcome
    Hospital mortality & & & o & & & & & \\
\end{longtable}


% Abréviations
\nomenclature[O]{SAPS}{simplified acute physiology score}
\nomenclature[O]{\(SOFA :\)}{Sequential Organ Failure Assessment}
\nomenclature[O]{\(qSOFA :\)}{quick Sequential Organ Failure Assessment}
\nomenclature[O]{\(SAPS II :\)}{Simplified Acute Physiology Score II}
\nomenclature[O]{\(ICU :\)}{intensive care units}
\nomenclature[O]{\(APACHE :\)}{Acute Physiology and Chronic Health Evaluation}
\nomenclature[M]{\(MAP :\)}{{Mean Arterial Pressure}
\nomenclature[O]{\(SIRS :\)}{Systemic Inflammatory Response Syndrome}
\nomenclature[O]{\(MIMIC :\)}{Multiparameter Intelligent Monitoring in Intensive Care}
\nomenclature[0]{MGP-TCN}{Multi-task Gaussian Process Adapters }
\nomunit{\SI{}{mmHg}}}

\newpage
\printnomenclature  % liste des abréviations

\newpage
\bibliographystyle{plain}
\bibliography{bibliographie}
\end{document}